\begin{abstract}

% This paper investigates an approach of concurrent learners, where the credit assignment problem is specially addressed (How to divide team reward among the individuals?). Moreover, the investigated approach is focussed on the dynamics of learning (How to cope with the problem of coadaptation?) in an instance of the homogeneous and (non-) communicating predator/prey scenario. To compensate the dynamic nature of such a scenario, a new algorithm is developed \emph{(SXCS)} by modelling the reward function according to an heuristic with high performance and by using memory to record and analyze the movements and the history of function's past values. 

Learning classifier systems (LCSs) are rule-based evolutionary reinforcement learning systems. Today, especially variants of Wilson's \emph{extended classifier system (XCS)} are widely applied for machine learning. Despite their widespread application, LCSs have drawbacks, e.\,g., in multi-learner scenarios, since the \emph{Markov property} is not fulfilled. % In such scenarios, typical challenges of research are: How to divide a team reward among the individuals and how to cope with the problem of parallel learning agents (coadaptation)?

In this paper, LCSs are investigated in an instance of the generic homogeneous and non-communicating predator/prey scenario. A group of predators collaboratively observe a (randomly) moving prey as long as possible, where each predator is equipped with a single, independent XCS. Results show that improvements in learning are achieved by cleverly adapting a \emph{multi-step} approach to the characteristics of the investigated scenario. Firstly, the environmental reward function is expanded to include sensory information. Secondly, the learners are equipped with a memory to store and analyze the history of local actions and given payoffs. % TODO evtl events erw�hnen

% The predator/prey domain is an appropriate example that has successfully been studied in a variety of instantiations. It does not serve as a complex real world domain, but as a test scenario for demonstrating and evaluating manifold research ideas. 

\end{abstract}
